\documentclass[conference]{IEEEtran}
\IEEEoverridecommandlockouts
\usepackage[dvipsnames]{xcolor}
\usepackage{enumitem}
\usepackage{balance}
\usepackage{pifont}
\usepackage{amsmath}
\usepackage{array}
\usepackage{listings}
\usepackage{microtype}
\usepackage{url}
\urlstyle{same}
\usepackage[hidelinks,bookmarks=false]{hyperref}

\lstset{
  basicstyle=\ttfamily\footnotesize,
  numbers=left,
  numberstyle=\tiny,
  frame=single,
  showstringspaces=false,
  breaklines=true,
  keywordstyle=\color{blue},
  commentstyle=\color{gray},
  stringstyle=\color{orange}
}

\newcommand{\cmark}{\ding{51}}
\newcommand{\xmark}{\ding{55}}

\title{{\textmu}Touch: Enabling Accurate, Lightweight Self-Touch Sensing with Passive Magnets\\(Artifact Guide)}

\author{
\IEEEauthorblockN{
Siyuan~Wang\IEEEauthorrefmark{1},
Ke~Li\IEEEauthorrefmark{2},
Jingyuan~Huang\IEEEauthorrefmark{1},
Jike~Wang\IEEEauthorrefmark{1},
Cheng~Zhang\IEEEauthorrefmark{2},
Alanson~Sample\IEEEauthorrefmark{3},
Dongyao~Chen\IEEEauthorrefmark{1}}
\IEEEauthorblockA{\IEEEauthorrefmark{1}Shanghai Jiao Tong University, China}
\IEEEauthorblockA{\IEEEauthorrefmark{2}Cornell University, USA}
\IEEEauthorblockA{\IEEEauthorrefmark{3}University of Michigan, USA}
\IEEEauthorblockA{Email: \{wsy0227, aquamarine\_indigo, jikewang, chendy\}@sjtu.edu.cn\\
\{kl975, chengzhang\}@cornell.edu;\ apsample@umich.edu}
}

\begin{document}
\maketitle
\sloppy

\section*{Quick Links}
\begin{center}
\begin{tabular}{@{}p{0.92in}>{\raggedright\arraybackslash}p{2.35in}@{}}
Code repo & \href{https://github.com/Wangmerlyn/muTouch}{github.com/Wangmerlyn/muTouch} \\
MagX base & \href{https://github.com/dychen24/magx}{github.com/dychen24/magx} \\
Models & Release tag \texttt{backup/3\_dim-models-20260121} on GitHub Releases (Assets) \\
PCB sources & \texttt{pcb/} (muTouch Altium project; legacy \texttt{Magway.*} filenames) \\
\end{tabular}
\end{center}

\section{Scope}
This guide describes the artifact supporting \textmu Touch: hardware (muTouch PCB + magnets) and software (BLE data collection, semi-supervised classifier). It targets reviewers who want to install, run, and validate the pipeline. Component specs and how they are verified: hardware specs and outputs in \texttt{pcb/README.md} (used in Section~\ref{sec:setup}); firmware/data-collection specs and expected signals in \texttt{Codes/read\_raw\_ble/README.md} (used in Sections~\ref{sec:setup}--\ref{sec:running}); project overview and folder purposes (including \texttt{artifact/}) in \texttt{README.md}.

\section{Notation}
\label{sec:notation}
Paths are relative to the repo root unless stated; runtime scripts assume \texttt{Codes/} as the working directory.

\section{Bill of Materials \& Requirements}
\subsection{Hardware (minimal)}
\begin{itemize}[leftmargin=*]
  \item muTouch PCB (Altium project in \texttt{pcb/}; assembled board; filenames use legacy \texttt{Magway.*}). PCB design by Xiaomeng Chen.
  \item 1\textendash2 passive N52 grade magnets (6--8\,mm recommended).
  \item Host laptop: Ubuntu 20.04+ or macOS 12+, 4-core CPU, $\geq$8\,GB RAM, BLE 4.0+ adapter.
  \item Optional: BLE USB dongle (if desktop lacks BLE).
\end{itemize}
\subsection{Software}
\begin{itemize}[leftmargin=*]
  \item Python 3.10; Conda virtual env named \texttt{muTouch} recommended.
  \item Git with submodules; CMake/Make only if you rebuild the C++ solver.
  \item Dependencies installed via \texttt{pip install -r Codes/requirements.txt}.
  \item LaTeX/PDF tools only needed to rebuild this document.
\end{itemize}

\section{Obtaining the Artifact}
\begin{enumerate}[leftmargin=1.2em]
  \item Clone the repository (now public):\\
  \begin{lstlisting}[language=bash,numbers=none]
git clone --recurse-submodules git@github.com:Wangmerlyn/muTouch.git
# HTTPS fallback:
# https://github.com/Wangmerlyn/muTouch.git
  \end{lstlisting}
  If you hit SSH ``permission denied'', use the HTTPS command above or ensure your SSH key is added to GitHub.
\item Activate env:\\
  \begin{lstlisting}[language=bash,numbers=none]
conda create -n muTouch python=3.10
conda activate muTouch
  \end{lstlisting}
  You should see the shell prompt prefixed with \texttt{(muTouch)} after activation; this isolates dependencies for repeatability.
\item Install deps:\\
  \begin{lstlisting}[language=bash,numbers=none]
pip install -r Codes/requirements.txt
  \end{lstlisting}
  Expect a clean install without errors; rerun from the repo root if paths are not found.
\item Models: snapshot tag \texttt{backup/3\_dim-models-20260121}.\\ Download binaries from GitHub Releases (Assets).
\item Set working directory for runtime scripts to \texttt{Codes/}:\\
  \begin{lstlisting}[language=bash,numbers=none]
cd Codes
  \end{lstlisting}
\end{enumerate}

\section{Setup \& Configuration}
\label{sec:setup}
\begin{enumerate}[leftmargin=1.2em]
  \item Flash firmware:\\
  {\scriptsize\texttt{Codes/Arduino/bleReadMultiple/bleReadMultiple.ino}} in Arduino IDE;\\
  select Feather nRF52; upload.
  \item Find BLE address: \texttt{python read\_raw\_ble/find\_device.py} (copy device MAC/UUID).
  \item Hardcode BLE address: edit the \texttt{address = ``...''} line near the bottom of\\
  \texttt{read\_raw\_ble/read\_sensor.py}, \texttt{read\_raw\_ble/read\_sensor\_real.py}, and (if used)\\ \texttt{read\_raw\_ble/read\_sensor\_real\_classifier.py}.
  \item Calibration capture (run inside \texttt{Codes/}):\\
  \begin{lstlisting}[language=bash,numbers=none]
python read_raw_ble/read_sensor.py
  \end{lstlisting}
  Do a brief figure-8 motion away from metal surfaces; CSVs are saved under \texttt{datasets/}.
  \item Offsets/scales: place the latest files named \texttt{offset-*.npy} and \texttt{scale-*.npy} in \path{calibration_files/}. The scripts automatically load the newest files with those prefixes.
  \item Models: ensure \texttt{read\_raw\_ble/models/} holds the downloaded checkpoint set if you need pretrained classifiers.
\end{enumerate}

\section{Running the Artifact}
\label{sec:running}
\subsection{Data capture}
\noindent Captures raw magnetometer streams over BLE with the latest calibration, writing timestamped CSVs under \texttt{datasets/}.
\begin{lstlisting}[language=bash,numbers=none]
python read_raw_ble/read_sensor_real.py
\end{lstlisting}

\subsection{Real-time classification}
\noindent Runs the live classifier; expect gesture labels printed to console and logged in \texttt{datasets/}.
\begin{lstlisting}[language=bash,numbers=none]
python read_raw_ble/read_sensor_real_classifier.py
\end{lstlisting}
Ensure the script uses the latest \texttt{offset-*}, \texttt{scale-*}, and model files (picked automatically if they sit in \texttt{calibration\_files/} and \texttt{read\_raw\_ble/models/}).

\subsection{Expected outcomes}
\begin{itemize}[leftmargin=1.2em]
  \item Face-touching: $\approx$93\% accuracy (8 gestures) with 3\,s fine-tuning/user.
  \item Scratch detection: $\approx$95\% accuracy (9 gestures) with 3\,s fine-tuning/user.
  \item Real-time loop maintains $>$30\,Hz inference on a laptop CPU.
\end{itemize}

\section{Reproducibility Checklist}
\begin{itemize}[leftmargin=*]
  \item \textbf{Hardware reproducible}: PCB sources + BOM (muTouch; files named \texttt{Magway.*} for compatibility) included.
  \item \textbf{Software reproducible}: All scripts + TS2Vec submodule; pinned deps in \texttt{Codes/requirements.txt}.
  \item \textbf{Data}: Calibration and small demo runs can be generated locally; full datasets are participant-specific and not included.
  \item \textbf{Pretrained models}: Provided via GitHub tag \texttt{backup/3\_dim-models-20260121}.
\end{itemize}

\section{Troubleshooting}
\begin{itemize}[leftmargin=*]
  \item BLE not found: retry \texttt{find\_device.py}; check power and pairing blocks; use BLE dongle.
  \item Drifting predictions: recalibrate sensors; ensure distance from large metal; re-run offset/scale.
  \item Import errors: confirm submodule init (\texttt{git submodule update --init --recursive}) and Python path from repo root.
\end{itemize}

\section{For Complete Instructions}
Due to page limits, this PDF is a concise checklist. For full, continuously updated steps and tips, please refer to the repository \texttt{README.md} and the Markdown guide (best viewed via GitHub preview): \href{https://github.com/Wangmerlyn/muTouch/blob/main/artifact/ARTIFACT_GUIDE.md}{\texttt{artifact/ARTIFACT\_GUIDE.md}}. Use those as the authoritative references for reproduction and further code tweaks.

\section{Notes on Prior Work}
The project builds on MagX (MobiCom'21) codebase for magnetic sensing~\cite{mag_x}. This artifact extends it to self-touch sensing and includes updated PCB by Xiaomeng Chen.

\balance
\bibliographystyle{IEEEtran}
\bibliography{artifact_refs}
\end{document}
