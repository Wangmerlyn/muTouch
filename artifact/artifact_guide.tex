\documentclass[10pt,conference]{IEEEtran}
\IEEEoverridecommandlockouts
\usepackage{hyperref}
\usepackage{enumitem}
\usepackage{balance}
\usepackage{pifont}
\usepackage{amsmath}

\newcommand{\cmark}{\ding{51}}
\newcommand{\xmark}{\ding{55}}

\title{\textmu Touch Artifact Guide}

\author{
  Siyuan Wang, Ke Li, Jingyuan Huang, Jike Wang, Cheng Zhang, Alanson Sample, Dongyao Chen\\
  (Artifacts correspond to the PerCom'26 paper ``\textmu Touch: Enabling Accurate, Lightweight Self-Touch Sensing with Passive Magnets'')
}

\begin{document}
\maketitle

\section{Scope}
This guide describes the artifact supporting \textmu Touch: hardware (Magway PCB + magnets) and software (BLE data collection, semi-supervised classifier). It targets reviewers who want to install, run, and validate the pipeline.

\section{Bill of Materials \& Requirements}
\subsection{Hardware (minimal)}
\begin{itemize}[leftmargin=*]
  \item Magway PCB (Altium project in \texttt{pcb/}; assembled board). PCB design by Xiaomeng Chen.
  \item 1\textendash2 passive N52 grade magnets (6--8\,mm recommended).
  \item Host laptop: Ubuntu 20.04+ or macOS 12+, 4-core CPU, $\geq$8\,GB RAM, BLE 4.0+ adapter.
  \item Optional: BLE USB dongle (if desktop lacks BLE).
\end{itemize}
\subsection{Software}
\begin{itemize}[leftmargin=*]
  \item Python 3.10; Conda recommended.
  \item Git with submodules; CMake/Make (only if rebuilding C++ solver).
  \item Dependencies from \texttt{pip install -e .[dev]}.
  \item Latex/PDF tools not required for runtime; only for this guide.
\end{itemize}

\section{Obtaining the Artifact}
\begin{enumerate}[leftmargin=*]
  \item Clone the repository (now public):\\
  \texttt{git clone --recurse-submodules git@github.com:Wangmerlyn/muTouch.git}\\
  (HTTPS alternative for CI: \texttt{https://github.com/Wangmerlyn/muTouch.git})
  \item Activate env: \texttt{conda create -n muTouch python=3.10; conda activate muTouch}
  \item Install deps: \texttt{pip install -e .[dev]; pre-commit install} (optional for lint).
  \item Models: a full model snapshot is tagged at \texttt{backup/3\_dim-models-20260121}. Download from GitHub Releases or checkout tag if needed for offline use.
\end{enumerate}

\section{Setup \& Configuration}
\begin{enumerate}[leftmargin=*]
  \item Flash firmware: open \texttt{Codes/Arduino/bleReadMultiple/bleReadMultiple.ino} in Arduino IDE, target Bluefruit nRF52 Feather, and upload.
  \item Find BLE address: \verb|python Codes/read_raw_ble/find_device.py|; copy device MAC/UUID.
  \item Calibration: \verb|python Codes/read_raw_ble/read_sensor.py --addr <BLE_ADDR> --out calibration.npy|; perform figure-8 motions away from metal.
  \item Offsets/scales: place generated \verb|offset-*|, \verb|scale-*| in \verb|calibration_files/| (or update paths in the scripts).
  \item Models: ensure \verb|Codes/read_raw_ble/models/| contains the downloaded checkpoint set if you need pretrained classifiers.
\end{enumerate}

\section{Running the Artifact}
\subsection{Data capture}
\begin{verbatim}
python Codes/read_raw_ble/read_sensor_real.py --addr <BLE_ADDR>
\end{verbatim}
Outputs timestamped CSVs under \texttt{datasets/}.

\subsection{Real-time classification}
\begin{verbatim}
python Codes/read_raw_ble/read_sensor_real_classifier.py --addr <BLE_ADDR>
\end{verbatim}
Ensure the script points to the latest \texttt{offset-*}, \texttt{scale-*}, and model files. Console prints detected gesture labels; logs are saved under \texttt{datasets/}.

\subsection{Expected outcomes}
\begin{itemize}[leftmargin=*]
  \item Face-touching: $\approx$93\% accuracy (8 gestures) with 3\,s fine-tuning/user.
  \item Scratch detection: $\approx$95\% accuracy across 12 participants.
  \item Real-time loop maintains $>$30\,Hz inference on a laptop CPU.
\end{itemize}

\section{Reproducibility Checklist}
\begin{itemize}[leftmargin=*]
  \item \textbf{Hardware reproducible}: PCB sources + BOM (Magway) included.
  \item \textbf{Software reproducible}: All scripts + TS2Vec submodule; pinned deps in \texttt{Codes/requirements.txt}.
  \item \textbf{Data}: Calibration and small demo runs can be generated locally; full datasets are participant-specific and not included.
  \item \textbf{Pretrained models}: Provided via GitHub tag \texttt{backup/3\_dim-models-20260121}.
\end{itemize}

\section{Troubleshooting}
\begin{itemize}[leftmargin=*]
  \item BLE not found: retry \texttt{find\_device.py}; check power and pairing blocks; use BLE dongle.
  \item Drifting predictions: recalibrate sensors; ensure distance from large metal; re-run offset/scale.
  \item Import errors: confirm submodule init (\texttt{git submodule update --init --recursive}) and Python path from repo root.
\end{itemize}

\section{Time Budget for Reviewers}
\begin{itemize}[leftmargin=*]
  \item Setup environment: 10--15 minutes.
  \item Flash firmware + calibration: 15 minutes.
  \item Run live classification demo: 5 minutes.
\end{itemize}

\section{Notes on Prior Work}
The project builds on MagX (MobiCom'21) codebase for magnetic sensing; source: \url{https://github.com/dychen24/magx}. This artifact extends it to self-touch sensing and includes updated PCB by Xiaomeng Chen.

\balance
\end{document}
